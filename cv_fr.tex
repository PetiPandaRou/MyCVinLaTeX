% FortySecondsCV LaTeX template
% Copyright © 2019-2020 René Wirnata <rene.wirnata@pandascience.net>
% Licensed under the 3-Clause BSD License. See LICENSE file for details.
%
% Please visit https://github.com/PandaScience/FortySecondsCV for the most
% recent version! For bugs or feature requests, please open a new issue on
% github.
%
% Contributors:
% https://github.com/PandaScience/FortySecondsCV/graphs/contributors
%
% Attributions
% ------------
% * fortysecondscv is based on the twentysecondcv class by Carmine Spagnuolo
%   (cspagnuolo@unisa.it), released under the MIT license and available under
%   https://github.com/spagnuolocarmine/TwentySecondsCurriculumVitae-LaTex
% * further attributions are indicated immediately before corresponding code


%-------------------------------------------------------------------------------
%                             ADDITIONAL PACKAGES
%-------------------------------------------------------------------------------
\documentclass[
	a4paper,
	% 9pt,
	% sidesectionsize=Large,
	% showframes,
	% vline=2.2em,
	maincolor=re,
	sidecolor=bg-2,
	sidetextcolor=tex-3,
	sectioncolor=re,
	subsectioncolor=tex-2,
	itemtextcolor=tex-3,
	% sidebarwidth=0.4\paperwidth,
	% topbottommargin=0.03\paperheight,
	% leftrightmargin=20pt,
	profilepicsize=3.5cm,
	profilepicborderwidth=2pt,
	% profilepicstyle=profilecircle,
	% profilepiczoom=1.0,
	% profilepicxshift=0mm,
	% profilepicyshift=0mm,
	% profilepicrounding=1.0cm,
	% logowidth=4.5cm,
	% logospace=5mm,
	% logoposition=before,
	% sidebarplacement=right,
	datecolwidth=0pt,
]{fortysecondscv}

% fine tune line spacing
% \usepackage{setspace}
% \setstretch{1.1}

% improve word spacing and hyphenation
\usepackage{microtype}
\usepackage{ragged2e}

% uncomment in case you don't want any hyphenation
\usepackage[none]{hyphenat}

% take care of proper font encoding
\ifxetexorluatex
	\usepackage{fontspec}
	\defaultfontfeatures{Ligatures=TeX}
	% \newfontfamily\headingfont[Path=fonts/]{segoeuib.ttf} % use local font
\else
	\usepackage[utf8]{inputenc}
	\usepackage[T1]{fontenc}
\fi

% use a sans serif font as default
\usepackage[sfdefault]{ClearSans}
% \usepackage[sfdefault]{noto}

% multi-language CV XeLaTeX and polyglossia (should also work with LuaLaTeX)
% NOTE: breaks \pointskill, \membership and some spacings
% \ifxetexorluatex
% 	\usepackage{polyglossia}
% 	\newfontfamily\arabicfontsf[Script=Arabic,Scale=1.5]{Amiri}
% 	\newfontfamily\englishfontsf{Clear Sans}
% 	\setmainfont{Amiri}
% 	\setdefaultlanguage{arabic}
% 	\setotherlanguage{english}
% \fi

% enable mathematical syntax for some symbols like \varnothing
\usepackage{amssymb}

% bubble diagram configuration
\usepackage{smartdiagram}
\smartdiagramset{
	% default font size is \large, so adjust to harmonize with sidebar layout
	bubble center node font = \footnotesize,
	bubble node font = \footnotesize,
	% default: 4cm/2.5cm; make minimum diameter relative to sidebar size
	bubble center node size = 0.4\sidebartextwidth,
	bubble node size = 0.25\sidebartextwidth,
	distance center/other bubbles = 1.5em,
	% set center bubble color
	bubble center node color = maincolor!70,
	% define the list of colors usable in the diagram
	set color list = {maincolor!10, maincolor!40,
	maincolor!20, maincolor!60, maincolor!35},
	% sets the opacity at which the bubbles are shown
	bubble fill opacity = 0.8,
}

%-------------------------------------------------------------------------------
%                            PERSONAL INFORMATION
%-------------------------------------------------------------------------------
%% mandatory information
% your name
\cvname{Julia BUCHNER}
% job title/career
\cvjobtitle{Senior Product Manager}

%% optional information
% profile picture
\cvprofilepic{pics/profile.png}

% NOTE: ordering in sidebar will mimic the following order
% date of birth
% \cvbirthday{3 Septembre 1986}
% short address/location, use \newline if more than 1 line is required
\cvaddress{Montpellier (34)}
% phone number
\cvphone{+33 (0)6 47 93 56 32}
% personal website
\cvsite{https://julia.buchner.fr}
% email address
\cvmail{julia@buchner.fr}
% pgp key
% \cvkey{4096R/FF00FF00}{0xAABBCCDDFF00FF00}
% any other custom entry
% \cvcustomdata{\faFlag}{Chinese}

%-------------------------------------------------------------------------------
%                              SIDEBAR 1st PAGE
%-------------------------------------------------------------------------------
% add more profile sections to sidebar on first page
\addtofrontsidebar{

	\profilesection{Savoir faire}

		\chartlabel{Product Management}

		\wheelchart{3.7em}{1.5em}{%
		20/3em/maincolor!10/{UX,UI et User Flow},
		20/3em/maincolor!50/Roadmap,
		20/4em/maincolor!20/Discovery,
		20/3em/maincolor!40/Data,
		20/3em/maincolor!20/{Communi-\newline cation}
		}

		\chartlabel{Divers}

		\skill{\faFootballBall}{Méthodes Agile \\(SCRUM, Shape Up, Kanban,...)}
		\skill{\faChartBar}{Analyse de données \\(Datadog, Metabase, SQL,...)}
		\skill{\faPenNib}{Rédaction documentaire}

	\profilesection{Savoir être}
		\pointskill{\faSitemap}{Organisation et rigueur}{4}[4]
			\skill[0em]{}{Forte des mes expériences en Support et en QA, je sais 
            optimiser mon temps de travail et communiquer en conséquent.}
		\pointskill{\faFile}{Capacité d'analyse}{4}[4]
			\skill[0em]{}{Travailler sur des problèmes fait partie de ma vie 
            d'ingénieur. Je suis capable de concevoir et mettre en place des 
			solutions simples pour résoudre des problèmes complexes.}
		\pointskill{\faUserFriends}{Esprit d'équipe}{3}[4]
			\skill[0em]{}{Motivée et volontaire, je sais m'intégrer dans une 
            équipe et travailler en collaboration avec tout le monde.}
}

%-------------------------------------------------------------------------------
%                              SIDEBAR 2nd PAGE
%-------------------------------------------------------------------------------
\addtobacksidebar{
	\graphicspath{{pics/flags/}}

	%\\profilesection{Projets IT}
	%\\chartlabel{Environnements}

	%\wheelchart{3.7em}{2em}{%
	%\40/3em/maincolor!50/Agile,
	%\15/4em/maincolor!15/Windows,
	%\15/3em/maincolor!30/Unix,
	%\15/3em/maincolor!20/macOS
	%\}

	%\\chartlabel{Softwares \& Languages}

	%\\wheelchart{3.7em}{2em}{%
	%\20/3em/maincolor!40/SQL,
	%\20/3em/maincolor!15/Powershell,
	%\30/4em/maincolor!40/Ordonnanceurs,
	%\10/3em/maincolor!20/HTML / CSS,
	%\15/3em/maincolor!55/\LaTeX~
	%\}

	\profilesection{Langues}
        \pointskill{\flag{FR.png}}{French}{5}
        \pointskill{\flag{GB.png}}{English}{4}
        \pointskill{\flag{IT.png}}{Italien}{1}

	\sidesection{Réseaux}
		\begin{icontable}{1.5em}{1em}
			\social{\faLinkedin}
				{https://www.linkedin.com/in/juliabuchner/}
				{linkedin.com/in/juliabuchner}
			\social{\faGithub}
				{https://github.com/PetiPandaRou}
				{github.com/PetiPandaRou}
		\end{icontable}

	\profilesection{Autres activités}
        \skill{\faBookReader}{Lecture}
            \skill[0em]{}{Grande lectrice, je lis de tout, avec une légère 
            préférence pour la science-fiction, la bande dessinée et les mangas.}
        \skill{\faGamepad}{Jeux vidéo}
            \skill[0em]{}{Tombée dans la marmite toute petite, j'adore les jeux 
			d'énigme et les récits narratifs, comme les jeux de Dontnod et 
			Quantic Dream. En ce moment, je parcours le monde de Phoenix Wright.}
		\skill{\faDice}{Jeux de rôle sur table}
            \skill[0em]{}{Depuis mon adolescence, je pratique le jeux de rôle
			sur table. En ce moment, je participe à plusieurs campagnes dans 
			l'univers de Donjons \& Dragons.}
        \skill{\faRunning}{Escrime}
            \skill[0em]{}{Pendant 5 ans, j'ai pratiqué de l'escrime sportive, 
            artistique ainsi que du sabrolaser. Je garde une activité sportive
			régulière.}
        \skill{\faTrain}{Voyage}
            \skill[0em]{}{Pour mes études, j'ai vécu une année en Chine et 
            réalisé de nombreux évènements en Europe pour mes activités 
            bénévoles.}
        \skill{\faCameraRetro}{Photographie}
            \skill[0em]{}{J'ai aiguisé mon sens de l'observation durant mes 
            différents voyages.}
	
}

%-------------------------------------------------------------------------------
%                         TABLE ENTRIES RIGHT COLUMN
%-------------------------------------------------------------------------------
\begin{document}
\pagecolor{bg}
\makefrontsidebar

\section{Résumé}
\vspace*{-15px} 
\begin{cvtable} 
	\cvitem{}{}{}{
		Product Manager expérimentée avec plus de 5 ans d'expérience dans la 
		gestion de produits techniques, la définition de stratégies produit et 
		le leadership d'équipes agiles, je cherche à construire des outils 
		innovants et des expériences utilisateurs exceptionnelles. 
		\newline\newline
		Secteur bancaire, assurance, automatisation logicielle, cloud gaming, 
		cybersécurité et e-commerce : j'ai eu l'opportunité de travailler dans 
		des secteurs professionnels nombreux et variés. Mon approche repose sur 
		l’écoute des besoins utilisateurs, la mise en œuvre de solutions 
		efficaces et évolutives, alliant performance, simplicité et amélioration 
		continue.
		}
\end{cvtable}

\cvsection{Expériences professionnelles}
\begin{cvtable}
	\cvitem{}{Head of Product - Lengow}
	{Janv. 2025 - Mars 2025 (3 mois)}{
		• Management d’une équipe de 5 Product Managers et 1 Product 
		Designer\newline
	 	• Préparation de la roadmap Produit sur l’ensemble de la plateforme
	 	(4 produits au total)\newline
	 	• Animation de la découverte produit avec entretiens utilisateurs, 
	 	étude concurrentielle et travail sur les métriques\newline
	 	• Rédaction de spécifications fonctionnelles et techniques (Go-To-Market 
		Strategy, Product Requirements Documents, Implementation 
		Strategy,...)\newline
	 	• Organisation de la livraison avec animation des rituels agiles (daily, 
		sprint planning, retro, roadmap,...)\newline
	 	• Documentation des processes et formation de l’équipe\newline
	 	• Communication entre les différentes équipes en interne (Dev, 
		Commerciales, Support) ainsi que les clients\newline
	}
	\cvitem{}{Lead Product Manager - Lengow}
	{Fév. 2024 -- Déc. 2024 (11 mois)}{
		• Management d’une équipe de 4 Product Managers et 1 Product Designer\newline
		• Construction de la stratégie sur deux produits de la plateforme\newline
		• Animation de la découverte produit avec entretiens utilisateurs, étude concurrentielle et travail sur les métriques\newline
		• Rédaction de spécifications fonctionnelles et techniques\newline
		• Organisation de la livraison avec animation des rituels agiles (daily, retro, roadmap,...)\newline
		• Documentation des processes\newline
		• Communication entre les différentes équipes en interne (Dev, Commerciales, Support) ainsi que les clients\newline
	}
    \cvitem{}{Product Manager - GitGuardian}
	{Nov. 2021 -- Oct. 2023 (2 ans)}{
		• Construction de la stratégie sur le produit auto-hébergé 
		GitGuardian\newline
		• Animation de la découverte produit avec entretiens utilisateurs,
		rédactions de MVP et travail sur les métriques\newline
		• Rédaction de spécifications fonctionnelles et techniques\newline
		• Organisation de la livraison avec animation des rituels agiles
		(daily, sprint planning, retro, roadmap,...)\newline
		• Documentation des processes et du produit\newline
		• Communication entre les différentes équipes en interne (Dev, QA, Commerciales) ainsi que les clients\newline
	}
	\cvitem{}{Incident Manager \& Product Owner - Shadow}
	{Mai 2020 -- Oct. 2021 (1 an et 5 mois)}{ }
	\cvitem{}{Technical Support Engineer - Shadow}
	{Avr. 2019 -- Avr. 2020 (1 an)}{ }
\end{cvtable}

\cvsection{Formations \& Certifications}
\begin{cvtable}
	\cvitem{}{Design System Expertise}{2024}
		{Construire, maximiser et passer à l'échelle l’adoption d’un Design
		System\newline}
	\cvitem{}{ITIL Foundation v3}{2017}
		{Management et qualité des systèmes d’information et services informatiques\newline}
	\cvitem{}{Science de l'Informatique, équivalent Ingénieur - SUPINFO}{2006 -- 2011}
		{Sujet de mémoire: Mise en place d'une gestion de projet SCRUM au sein 
        d'une équipe décentralisée.\newline}
\end{cvtable}

\newpage
\makebacksidebar
% \newgeometry{
% 	top=\topbottommargin,
% 	bottom=\topbottommargin,
% 	right=\leftrightmargin,
% 	left=\leftrightmargin
% }
\cvsection{Expériences Associatives}
\begin{cvtable}
	\cvitem{}{Rédactrice - Les sans pagEs}{2024 -- Maintenant (1 an)}{
		\textit{MISSION: Le projet les sans pagEs est né du besoin de combler le fossé 
		et le biais de genre sur Wikipédia: en 2016, Wikipédia en français 
		compte 450 000 biographies d'hommes, contre 75 000 de femmes, soit 
		seulement 16,6\%. En 2024, ce chiffre est passé à 19,98\%.}
		\newline\newline
        • Rédaction et traduction d’articles\newline
        • Organisation événementielle\newline
    }
	\cvitem{}{Bénévole \& Trésorière - StartHer}{2010 -- 2017 (7 ans)}{
		\textit{MISSION: L'association StartHer, anciennement Girls in Tech 
		mettait en avant les femmes et leurs parcours dans la technologie afin 
		d'encourager la mixité et la diversité dans l’univers entrepreneurial.}
		\newline\newline
        • Gestion du système informatique de l’association\newline
        • Mise en place d’environnements de test dans le cadre d’un projet 
        d’évolution du site web\newline
        • Organisation événementielle et logistique\newline
        • Trésorerie\newline
        • Rédaction d’articles\newline
    }
	\cvitem{}{Bénévole - Mozilla Foundation}{2008 -- 2016 (8 ans)}{
		\textit{MISSION: La fondation Mozilla défend Internet comme une 
		ressource publique, ouverte et accessible à tous.}
		\newline\newline
        • Rédaction d’articles et de documentations fonctionnelles et techniques\newline
        • Organisation événementielle et animation de conférences\newline
        • Community Management\newline
        • Traduction et transcription\newline
        • Mascotte\newline
	}
	\cvitem{}{Bénévole - April}{22008 -- 2016 (8 ans)}{
		\textit{MISSION: L'association April travaille à la défense et à la 
		diffusion du Logiciel Libre et des standards ouverts..}
		\newline\newline
        • Rédaction d’articles et de documentations\newline
        • Organisation événementielle et animation de conférences\newline
        • Traduction et transcription\newline
    }
	\cvitem{}{Rédactrice - TechCrunch France}{2011 -- 2012 (1 an)}{ }
	\cvitem{}{Rédactrice - GeekInc}{2009 -- 2010 (1 an)}{ }
\end{cvtable}

\end{document}
