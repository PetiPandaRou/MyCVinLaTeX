% FortySecondsCV LaTeX template
% Copyright © 2019-2020 René Wirnata <rene.wirnata@pandascience.net>
% Licensed under the 3-Clause BSD License. See LICENSE file for details.
%
% Please visit https://github.com/PandaScience/FortySecondsCV for the most
% recent version! For bugs or feature requests, please open a new issue on
% github.
%
% Contributors:
% https://github.com/PandaScience/FortySecondsCV/graphs/contributors
%
% Attributions
% ------------
% * fortysecondscv is based on the twentysecondcv class by Carmine Spagnuolo
%   (cspagnuolo@unisa.it), released under the MIT license and available under
%   https://github.com/spagnuolocarmine/TwentySecondsCurriculumVitae-LaTex
% * further attributions are indicated immediately before corresponding code


%-------------------------------------------------------------------------------
%                             ADDITIONAL PACKAGES
%-------------------------------------------------------------------------------
\documentclass[
	a4paper,
	% 9pt,
	% sidesectionsize=Large,
	% showframes,
	% vline=2.2em,
	% maincolor=cvgreen,
	% sidecolor=gray!50,
	% sidetextcolor=green,
	% sectioncolor=red,
	% subsectioncolor=orange,
	% itemtextcolor=black!80,
	% sidebarwidth=0.4\paperwidth,
	% topbottommargin=0.03\paperheight,
	% leftrightmargin=20pt,
	% profilepicsize=4.5cm,
	% profilepicborderwidth=3.5pt,
	% profilepicstyle=profilecircle,
	% profilepiczoom=1.0,
	% profilepicxshift=0mm,
	% profilepicyshift=0mm,
	% profilepicrounding=1.0cm,
	% logowidth=4.5cm,
	% logospace=5mm,
	% logoposition=before,
	% sidebarplacement=right,
]{fortysecondscv}

% fine tune line spacing
% \usepackage{setspace}
% \setstretch{1.1}

% improve word spacing and hyphenation
\usepackage{microtype}
\usepackage{ragged2e}

% uncomment in case you don't want any hyphenation
\usepackage[none]{hyphenat}

% take care of proper font encoding
\ifxetexorluatex
	\usepackage{fontspec}
	\defaultfontfeatures{Ligatures=TeX}
	% \newfontfamily\headingfont[Path=fonts/]{segoeuib.ttf} % use local font
\else
	\usepackage[utf8]{inputenc}
	\usepackage[T1]{fontenc}
\fi

% use a sans serif font as default
\usepackage[sfdefault]{ClearSans}
% \usepackage[sfdefault]{noto}

% multi-language CV XeLaTeX and polyglossia (should also work with LuaLaTeX)
% NOTE: breaks \pointskill, \membership and some spacings
% \ifxetexorluatex
% 	\usepackage{polyglossia}
% 	\newfontfamily\arabicfontsf[Script=Arabic,Scale=1.5]{Amiri}
% 	\newfontfamily\englishfontsf{Clear Sans}
% 	\setmainfont{Amiri}
% 	\setdefaultlanguage{arabic}
% 	\setotherlanguage{english}
% \fi

% enable mathematical syntax for some symbols like \varnothing
\usepackage{amssymb}

% bubble diagram configuration
\usepackage{smartdiagram}
\smartdiagramset{
	% default font size is \large, so adjust to harmonize with sidebar layout
	bubble center node font = \footnotesize,
	bubble node font = \footnotesize,
	% default: 4cm/2.5cm; make minimum diameter relative to sidebar size
	bubble center node size = 0.4\sidebartextwidth,
	bubble node size = 0.25\sidebartextwidth,
	distance center/other bubbles = 1.5em,
	% set center bubble color
	bubble center node color = maincolor!70,
	% define the list of colors usable in the diagram
	set color list = {maincolor!10, maincolor!40,
	maincolor!20, maincolor!60, maincolor!35},
	% sets the opacity at which the bubbles are shown
	bubble fill opacity = 0.8,
}

%-------------------------------------------------------------------------------
%                            PERSONAL INFORMATION
%-------------------------------------------------------------------------------
%% mandatory information
% your name
\cvname{Julia BUCHNER}
% job title/career
\cvjobtitle{Product Manager}

%% optional information
% profile picture
\cvprofilepic{pics/profile.png}

% NOTE: ordering in sidebar will mimic the following order
% date of birth
% \cvbirthday{3 Septembre 1986}
% short address/location, use \newline if more than 1 line is required
\cvaddress{Montpellier (34)}
% phone number
\cvphone{+33 (0)6 47 93 56 32}
% personal website
\cvsite{https://julia.buchner.fr}
% email address
\cvmail{julia@buchner.fr}
% pgp key
% \cvkey{4096R/FF00FF00}{0xAABBCCDDFF00FF00}
% any other custom entry
% \cvcustomdata{\faFlag}{Chinese}

%-------------------------------------------------------------------------------
%                              SIDEBAR 1st PAGE
%-------------------------------------------------------------------------------
% add more profile sections to sidebar on first page
\addtofrontsidebar{

	\profilesection{Savoir faire}
		\skill{\faFootballBall}{SCRUM / Product Management}
		\skill{\faSitemap}{Gestion de projets}
		\skill{\faChartBar}{Analyse de problèmes}
		\skill{\faPenNib}{Rédaction documentaire}

	\profilesection{Savoir être}
		\pointskill{\faSitemap}{Organisation et rigueur}{4}[4]
			\skill[0em]{}{Forte des mes expériences en Support et en QA, je sais 
            optimiser mon temps de travail et communiquer en conséquent.}
		\pointskill{\faFile}{Capacité d'analyse}{4}[4]
			\skill[0em]{}{Travailler sur des problèmes fait partie de ma vie 
            d'ingénieur, notamment durant mes expériences en production et au 
            support. Je suis également capable de concevoir et mettre en place 
            des solutions complexes pour résoudre des problèmes.}
		\pointskill{\faGraduationCap}{Curieuse}{3}[4]
			\skill[0em]{}{Désireuse d'apprendre, je suis toujours en recherche 
            de nouvelles connaissances.}
		\pointskill{\faUserFriends}{Esprit d'équipe}{3}[4]
			\skill[0em]{}{Motivée et volontaire, je sais m'intégrer dans une 
            équipe et travailler en collaboration avec tout le monde.}
}

%-------------------------------------------------------------------------------
%                              SIDEBAR 2nd PAGE
%-------------------------------------------------------------------------------
\addtobacksidebar{
	\graphicspath{{pics/flags/}}

	%\\profilesection{Projets IT}
	%\\chartlabel{Environnements}

	%\wheelchart{3.7em}{2em}{%
	%\40/3em/maincolor!50/Agile,
	%\15/4em/maincolor!15/Windows,
	%\15/3em/maincolor!30/Unix,
	%\15/3em/maincolor!20/macOS
	%\}

	%\\chartlabel{Softwares \& Languages}

	%\\wheelchart{3.7em}{2em}{%
	%\20/3em/maincolor!40/SQL,
	%\20/3em/maincolor!15/Powershell,
	%\30/4em/maincolor!40/Ordonnanceurs,
	%\10/3em/maincolor!20/HTML / CSS,
	%\15/3em/maincolor!55/\LaTeX~
	%\}

	\profilesection{Langues}
        \pointskill{\flag{FR.png}}{French}{5}
        \pointskill{\flag{GB.png}}{English}{4}
        \pointskill{\flag{IT.png}}{Italien}{1}

	\profilesection{Autres activités}
        \skill{\faBookReader}{Lecture}
            \skill[0em]{}{Grande lectrice, je lis de tout, avec une légère 
            préférence pour la science-fiction, la bande dessinée et les mangas.}
        \skill{\faGamepad}{Jeux vidéo}
            \skill[0em]{}{Tombée dans la marmite toute petite, j'adore les jeux 
			d'énigme et les récits narratifs, comme les jeux de Dontnod et 
			Quantic Dream. En ce moment, je parcours le monde d'Elyos dans
			Fire Emblem Engage.}
		\skill{\faDice}{Jeux de rôle sur table}
            \skill[0em]{}{Depuis mon adolescence, je pratique le jeux de rôle
			sur table. En ce moment, je participe à plusieurs campagnes dans 
			l'univers de Donjons \& Dragons.}
        \skill{\faRunning}{Escrime}
            \skill[0em]{}{Pendant 5 ans, j'ai pratiqué de l'escrime sportive, 
            artistique ainsi que du sabrolaser. Je garde une activité sportive
			régulière.}
        \skill{\faTrain}{Voyage}
            \skill[0em]{}{Pour mes études, j'ai vécu une année en Chine et 
            réalisé de nombreux évènements en Europe pour mes activités 
            bénévoles.}
        \skill{\faCameraRetro}{Photographie}
            \skill[0em]{}{J'ai aiguisé mon sens de l'observation durant mes 
            différents voyages.}
}

%-------------------------------------------------------------------------------
%                         TABLE ENTRIES RIGHT COLUMN
%-------------------------------------------------------------------------------
\begin{document}

\makefrontsidebar

\cvsection{Expériences Professionnelles}
\begin{cvtable}[3]
    \cvitem{Nov. 2021 -- Oct. 2023\newline(2 ans)}{Product Manager - GitGuardian}{}
        {
			• Construction de la stratégie sur le produit auto-hébergé 
			GitGuardian\newline
			• Animation de la découverte produit avec entretiens utilisateurs,
			rédactions de MVP et travail sur les métriques\newline
			• Rédaction de spécifications fonctionnelles et techniques\newline
			• Organisation de la livraison avec animation des rituels agiles
			(daily, sprint planning, retro, roadmap,...)\newline
			• Documentation des processes et du produit\newline
			• Communication entre les différentes équipes en interne (Dev, QA, Commerciales) ainsi que les clients\newline
		}
    \cvitem{Mai 2020 -- Oct. 2021\newline (1 an et 5 mois)}{Incident Manager \& Product Owner - Shadow}{}
        {
            • Mise en place et amélioration des méthodologies d'escalade des
            incidents et problèmes\newline
            • Qualification, analyse et reproduction des escalades techniques
            \newline
            • Priorisation des problèmes au sein du backlog de développement
            \newline
            • Documentation des processes et du produit\newline
            • Rédaction de spécifications fonctionnelles et techniques\newline
            • Mise en place et analyse de métriques\newline
            • Communication entre les différentes équipes (Dev, QA, Support, Community Management) ainsi que les clients\newline
        }

    \cvitem{Avril 2019 -- Mai 2020\newline (1 an)}{Ingénieur Support Logiciel - Shadow}{}
        {
            • Mise en place des méthodologies d'escalade des incidents et de 
            problèmes\newline
            • Qualification, analyse et reproduction des escalades techniques
            \newline
            • Documentation des processes et du produit\newline
            • Communication entre les différentes équipes (Dev, Support, 
            \newline Community Management)\newline
        }
    \cvitem{Juin 2017 -- Mai 2019\newline (2 ans)}{Responsable Technique de Compte - CA Technologies}{}
        {
            • Support de niveau 3 sur l'ordonnanceur ONE Automation pour les 
            clients Grands Comptes\newline
            • Analyse et reproduction des problématiques\newline
            • Proposition de solution et assistance pour la mise en œuvre\newline
            • Assistance utilisateur (configuration, administration, 
            \newline amélioration,…)\newline
            • Rédaction fonctionnelle et technique\newline
        }
\end{cvtable}


\cvsection{Certifications}
\begin{cvtable}[1.5]
	\cvitem{2017}{ITIL Foundation v3}{}
		{Information Technology Infrastructure Library}
	\cvitem{2011}{Test of English for International Communication}{}
		{Score: 855 / 990\\}
\end{cvtable}


\cvsection{Formation}
\begin{cvtable}[1.5]
	\cvitem{2006 -- 2011}{Science de l'Informatique, équivalent Ingénieur}{SUPINFO}
		{Sujet de mémoire: Mise en place d'une gestion de projet SCRUM au sein 
        d'une équipe décentralisée.}
\end{cvtable}


\newpage
\makebacksidebar
% \newgeometry{
% 	top=\topbottommargin,
% 	bottom=\topbottommargin,
% 	right=\leftrightmargin,
% 	left=\leftrightmargin
% }
\cvsection{Expériences Associatives}
\begin{cvtable}[3]
	\cvitem{2010 -- 2017\newline (7 ans)}{Bénévole \& Trésorière - StartHer}{}
    {
        • Gestion du système informatique de l’association\newline
        • Mise en place d’environnements de test dans le cadre d’un projet 
        d’évolution du site web\newline
        • Organisation événementielle et logistique\newline
        • Trésorerie\newline
        • Rédaction d’articles\newline
        MISSION: Mise en avant des femmes et de leurs parcours dans la 
        technologie afin d'encourager la mixité et la diversité dans l’univers 
        entrepreneurial.\newline
    }
	\cvitem{2008 -- 2016\newline (8 ans)}{Bénévole - Mozilla Foundation}{}
    {    
        • Rédaction d’articles et de documentations fonctionnelles et \newline
        techniques\newline
        • Organisation événementielle et animation de conférences\newline
        • Community Management\newline
        • Traduction et transcription\newline
        • Mascotte\newline
        MISSION: Défense d’Internet comme une ressource publique, ouverte et 
        accessible à tous.\newline
	}
	\cvitem{2008 -- 2016\newline (8 ans)}{Bénévole - April}{}
    {
        • Rédaction d’articles et de documentations\newline
        • Organisation événementielle et animation de conférences\newline
        • Traduction et transcription\newline
        MISSION: Défense et diffusion du Logiciel Libre et des standards 
        ouverts.\newline
    }
	\cvitem{2011 -- 2012\newline (1 an)}{Rédactrice - TechCrunch France}{}{ }
	\cvitem{2009 -- 2010\newline (1 an)}{Rédactrice - GeekInc}{}{ }
\end{cvtable}

\end{document}
